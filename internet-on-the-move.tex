%% THIS IS A DRAFT OF THE INTERNET ON THE MOVE SURVEY
%% its based on a skeleton file demonstrating the use of IEEEtran.cls
%% (requires IEEEtran.cls version 1.7 or later) with an IEEE journal paper.
%% Info on how I set this up is avaliable at on my blog
%% http://etc-hosts.blogspot.co.uk/2012/11/latex-ieee-surveys-tutorials-way-pt-1.html

% Also note that the "draftcls" or "draftclsnofoot", not "draft", option
% should be used if it is desired that the figures are to be displayed in
% draft mode.

\documentclass[journal]{IEEEtran}


% *** CITATION PACKAGES ***
%
%\usepackage{cite}
% cite.sty was written by Donald Arseneau
% V1.6 and later of IEEEtran pre-defines the format of the cite.sty package
% \cite{} output to follow that of IEEE. Loading the cite package will
% result in citation numbers being automatically sorted and properly
% "compressed/ranged". e.g., [1], [9], [2], [7], [5], [6] without using
% cite.sty will become [1], [2], [5]--[7], [9] using cite.sty. cite.sty's
% \cite will automatically add leading space, if needed. Use cite.sty's
% noadjust option (cite.sty V3.8 and later) if you want to turn this off.
% cite.sty is already installed on most LaTeX systems. Be sure and use
% version 4.0 (2003-05-27) and later if using hyperref.sty. cite.sty does
% not currently provide for hyperlinked citations.
% The latest version can be obtained at:
% http://www.ctan.org/tex-archive/macros/latex/contrib/cite/
% The documentation is contained in the cite.sty file itself.



\begin{document}

% working title, please replace with better title
% you can use linebreaks \\ title
\title{Internet On The Move Survey}


% I dont know how you order author or how exactly are the authors 
% so I've gone for alphabetical ordering

\author{Jon Crowcroft, Heidi Howard, Arjuna Sathiaseelan, Narseo Valina} % <-this % stops a space
        
% adding any thanks using \thanks{ }% <-this % stops a space

% The paper headers
\markboth{Internet on the move survey DRAFT}%
{Internet on the move survey DRAFT}
% The only time the second header will appear is for the odd numbered pages
% after the title page when using the twoside option.


% make the title area
\maketitle


\begin{abstract}
%\boldmath
The abstract goes here.
\end{abstract}


% Note that keywords are not normally used for peerreview papers.
\begin{IEEEkeywords}
IEEEtran, journal, \LaTeX, paper, template.
\end{IEEEkeywords}



\section{Introduction}

% 
% Here we have the typical use of a "T" for an initial drop letter
% and "HIS" in caps to complete the first word.
\IEEEPARstart{T}{his} demo file is intended to serve as a ``starter file''
for IEEE journal papers produced under \LaTeX\ using
IEEEtran.cls version 1.7 and later.

 
\hfill January 11, 2007

\section{Global Scenario}

\section{European Scenario}
% countries with excellent coverage i.e. sweden, finland, south korea
% countries with poor coverage
% what were the factors that lead to these differences 
\subsection{Key Players}
\subsection{Mobile Broadband Penetration}
\subsection{Coverage}
\subsubsection{2G}
\subsubsection{3G}
\subsubsection{4G}
\subsection{Spectrum Allocation}
\subsection{Mobile Broadband Traffic}
\subsection{Wide area WiFi networks}
\subsection{Infrastructure}

\section{UK Scenario}
% Digital By Default
% Open Data Project
% Sale of Radio Spectum (http://www.pcpro.co.uk/news/377980/eu-bosses-free-up-more-radio-spectrum-for-4g)
% Data Communications Bill
% Broadband Delivery UK
% Wireless Commons Licience 
% Net Neutrality Debate 
% Using of smartphones
% Comparion of the coverage claimed by carriers and the reality
% Coverage of 3G/4G provided by UK's main carriers
% results of population survey about people without internet access


\section{Application Layer}
\subsection{Bonding}
%  i.e. Opengarden
\subsection{AAA}
% using RADIUS
% using Diameter
\subsection{Security}
% TLS/SSL
\subsection{DNS}
%DNSSEC
% .mobi TLD
\subsection{SIP}
% session initiation protocol
\subsection{Delay Tolerant Networks}

\section{Transport Layer}
\subsection{ multipath TCP }
% TCP-R
\subsection{ Multi-homed TCP }
\subsection{Stream Control Transmission Protocol (SCTP)}
\subsection{Datagram Congestion Control Protocol (DCCP)}

\section{Network Layer}
\subsection{Mobile IP}
\subsection{Host Identity Protocol (HIP)}
\subsection{IPsec}
\subsection{IPv4}
% QoS field
\subsection{Transition to IPv6}
\subsection{Hierarchical Mobile IPv6}
\subsection{Fast Mobile IPv6}
\subsection{Resource Reservation Protocol}
% for QoS and Traffic Engineering
\subsection{Cisco's Hot Standby Router Protocol (HSRP)}
\subsection{Virtual Router Redundancy Protocol (VRRP)}
\subsection{Less then best effect (scavanger class)}

\section{Data Link and Physical Layer}

\subsection{ Multiprotocol Label Switching (MPLS) for QoS }
\subsection{ Offloading/Onloading }
\subsection{ 3G Onloading (3GOL) }
\subsection{ Bonding 3G and Wifi }
\subsection{ Wifi (IEEE 802.11) }
\subsection{ WiMAX }
\subsection{ WiMax 2 }
\subsection{ Mobile WiMax (IEEE 802.16e) }
\subsection{ WiMax Advanced (IEEE 802.16m) }
\subsection{ LTE }
\subsection{ LTE Advance }
\subsection{ High-Speed Packet Access (HSPA) }
\subsection{ Evolved High-Speed Packet Access (HSPA+) }
\subsection{ 3rd Generation Partnership Project (3GPP) }
\subsection{ 4G } 
\subsection{ 5G }
\subsection{ WiFox }
\subsection{ WPA and WPA2 }
\subsection{ Mesh networking (IEEE 802.11s) }
\subsection{ Mobile Broadband Wireless Access (MBWA) (IEEE 802.20)}
\subsection{ Joint user multi-user beamforming }
\subsection{ MIMO transmitter }
\subsection{ Pocket Switched Networks }
\subsection{ Virtual LANs for QoS }


\section{Cloud and Virtualization}
\subsection{ Signposts }

\section{Multilayer}
\subsection{ LCD }
\subsection{ Eduroam/JANET european acedemic network }
\subsection{ measuring performace acheived }
\subsection{ BT FON }
\subsection{ Guifi }
\subsection{ PAWS }
\subsection{ Opportunistic Communication }
\subsection{ Home Location Register (HLR) }
\subsection{ Home Subscriber Service (HSS) }

\section{Conclusion}
The conclusion goes here.


% references section

% can use a bibliography generated by BibTeX as a .bbl file
% BibTeX documentation can be easily obtained at:
% http://www.ctan.org/tex-archive/biblio/bibtex/contrib/doc/
% The IEEEtran BibTeX style support page is at:
% http://www.michaelshell.org/tex/ieeetran/bibtex/
%\bibliographystyle{IEEEtran}
% argument is your BibTeX string definitions and bibliography database(s)
%\bibliography{IEEEabrv,../bib/paper}
%
% <OR> manually copy in the resultant .bbl file
% set second argument of \begin to the number of references
% (used to reserve space for the reference number labels box)
\begin{thebibliography}{1}

\bibitem{IEEEhowto:kopka}
H.~Kopka and P.~W. Daly, \emph{A Guide to \LaTeX}, 3rd~ed.\hskip 1em plus
  0.5em minus 0.4em\relax Harlow, England: Addison-Wesley, 1999.

\end{thebibliography}


\end{document}

% Before submitting one of us need to do the testflow test
% support page for this is at: http://www.michaelshell.org/tex/testflow/